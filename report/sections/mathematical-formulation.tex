%=====================
% Remove this after writing actual report
%=====================
%
%Maecenas sed ultricies felis. Sed imperdiet dictum arcu a egestas. 
%\begin{itemize}
%	\item Donec dolor arcu, rutrum id molestie in, viverra sed diam
%	\item Curabitur feugiat
%	\item turpis sed auctor facilisis
%	\item arcu eros accumsan lorem, at posuere mi diam sit amet tortor
%	\item Fusce fermentum, mi sit amet euismod rutrum
%	\item sem lorem molestie diam, iaculis aliquet sapien tortor non nisi
%	\item Pellentesque bibendum pretium aliquet
%\end{itemize}
%\blindtext
%
%Text requiring further explanation\footnote{Example footnote}.

%=====================
%  MATHEMATICS
%=====================
To accomplish our task, we implement a Support-Vector Machine. This method can be used to separate, with a line\footnote{A pivotal assumption in our application is that our data (the images of different digits) are indeed linearly separable. We will find that this is mostly true, but not 100\% so.}, two classes of points in a dataset with a separating hyperplane:
\begin{center}
$\vec{w}\cdot\vec{x} + b = 0$\end{center}
The two sets' vectors with the minimum distance to each other are defined as the support vectors, and assigned a $y_{i}$ shift of $1$ or $-1$ from the separating. These vectors form the two constraints on our optimal hyperplane:
\begin{center}
$\vec{w}\cdot\vec{x_0} + b \geq 1 $ , if $  \vec{y_i} = 1$

$\vec{w}\cdot\vec{x_1} + b \leq -1 $ , if $  \vec{y_i} = -1$
\end{center}
This form is called canonical, as we have chosen to define y $\in$ ${-1, 1}$. The choices of -1 and 1 are arbitrary, but result in the most efficient calculation in computer applications.

The optimal hyperplane is defined as the plane that results in zero error and that maximizes the distance from the support vectors. In our use of two canonical classes, this distance can be derived to be $\frac{1}{2}\vec{w}\cdot\vec{w}$, or $\frac{||\vec{w}||}{2}$, and this is shown in the Appendix.

Since we need to maximize this distance to find our optimal hyperplane (that is, the plane with the largest accuracy )