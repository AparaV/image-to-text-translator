%=====================
%  INTRODUCTION
%=====================
\lettrine[nindent=0em,lines=3]{C}{lassifying} text into the individual characters and digits commonly seen in the world around is easily accomplished with the human brain. This is because, Through repeated use, humans have been trained to comprehend these characters. In order for computers to accomplish this same task, they need to be trained as well. Since images can be represented as matrices of individual pixels, the natural processing method is through linear algebra. To realistically perform this task, we must use optimization to fit an imperfect image (found on MNIST) to our ideal models of our target classes (the digits 0-9).

Using Support-Vector Networks (or SVN), it is possible to identify these digits from an image that was sized beforehand. SVN has been previously studied before by people including Vladimir N. Vapnik. SVN works by taking a series support vectors, separating them into two classes, and using it to create an optimized hyperplane \cite{statistical-learning}. This would separate the support vectors into two regions with the largest possible margin between the hyperplane and both of the dataset. This creates a maximization problem between the two, following equation:
\begin{equation}
  \vec{w} \cdot \vec{x}-b=1
\end{equation}

Trough further implementation, it is possible to segment data into multiple groups. To do this we will use SVN to create hyperplanes that isolate the individual sets and use these to determine each new input. This will then be used to compute each of the digits as separate groups.  The MNIST dataset is especially useful as it gives a large amount of images that are already sized. This makes it so that we don't have to size the images, and it contains a large enough sample space to make an accurate model. This will also use a library that allows the use of SVN for computation \cite{scikit-learn}.

Text recognition is extremely useful to the real world. Using it, it is possible for machines to communicate with the world through written information. Programs have had success with using text recognition, including translation, organizing books, communicate with computers and more. Google has implemented this into its phones where it is possible to take text from a different language identify it and output the text translated into the readers language. Also, for books and old written information and make it more accessible. By placing this into a database it saves tedious and possibly difficult work. Also, it allows for people to communicate with phones and computers in a simpler fashion. Instead of using a regular keyboard, it becomes possible to communicate with voice or writing making it an easier process for the user.