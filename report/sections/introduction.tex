%=====================
%  INTRODUCTION
%=====================

\lettrine[nindent=0em,lines=2]{T}{he} problem of using computers to automatically classify images is a very difficult problem, and is, rightly, heavily studied. This is a particularly interesting question because, while computers can efficiently solve large problems that are difficult for humans, they struggle with tasks that humans find easy, such as identifying text in images. Over the past few decades, the broader classification problem has been tackled by mathematicians using varied and novel techniques. One of these techniques, originally discussed by Vladimir Vapnik in $1994$, are \textit{Support Vector Machines}\cite{statistical-learning}. This novel method approached the problem by constructing the \textit{optimal hyperplane} that separates data into two different classes.

For our project, we chose to investigate how we can use Support Vector Machines (SVM) to classify handwritten digits i.e., given an image containing a single digit (or not), how can we use SVMs to identify the digit. We chose this because, recognizing text in images has a wide array of applications from creating digital records of old texts to creating searchable images of directories, menus, etc. Owing to the complexity of the larger problem, we chose a subset - classifying single digits. For our project, we will be using the MNIST database\cite{mnist}. The MNIST database, compiled by LeCunn, et. al., consists of totally $70,000$ grayscale images of handwritten digits. The database was originally compiled for the purpose of building a classifier that would automatically recognizing and sorting postal mail by zipcodes.

The rest of the report is organized as follows: Section \textbf{\ref{section:math}} describes the mathematics, significance, and caveats of SVM; Section \textbf{\ref{section:experiments}} describes the experiments we performed on the MNIST database using SVM; Section \textbf{\ref{section:results}} discusses the results we got from our experiments in section \textbf{\ref{section:experiments}} and; Section \textbf{\ref{section:conclusion}} discusses the conclusions of our project, and provides some ideas for extending our project and future research.